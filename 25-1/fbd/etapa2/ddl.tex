\documentclass[12pt]{article}
\usepackage[utf8]{inputenc}
\usepackage[T1]{fontenc}
\usepackage{fontspec}
\usepackage{moresize}
\usepackage[brazil]{babel}
\usepackage{geometry}
\usepackage{xcolor}
\usepackage{listings}
\usepackage{titling}
\usepackage{hyperref}
\usepackage{titlesec}

\geometry{
    paper=a4paper,
    top=64pt,
    bottom=64pt,
    left=112pt,
    right=112pt
}

\setmainfont{Libertinus Serif}[
  UprightFont = *,
  ItalicFont = * Italic,
  BoldFont = * Bold,
  BoldItalicFont = * Bold Italic
]
\setmonofont{Fira Code}[
  Scale=MatchLowercase, Ligatures=TeX
]

\titlespacing*{\subsection}{0pt}{2em}{0.625em}

\definecolor{sky}{HTML}{0084d1}
\definecolor{fuchsia}{HTML}{c800de}
\definecolor{gray50}{HTML}{f9fafb}
\definecolor{gray400}{HTML}{99a1af}
\definecolor{gray900}{HTML}{101828}

\lstdefinestyle{sqlstyle}{
    language=SQL,
    basicstyle=\ttfamily\color{gray900},
    keywordstyle=\color{sky}\bfseries,
    stringstyle=\color{fuchsia},
    commentstyle=\color{gray400},
    backgroundcolor=\color{gray50},
    rulecolor=\color{gray50},
    frame=single,
    framesep=16pt,
    xleftmargin=16pt,
    xrightmargin=16pt,
    breaklines=true
}
\lstset{style=sqlstyle}

\NewDocumentCommand{\course}{m}{\gdef\thecourse{#1}}
\NewDocumentCommand{\authors}{m}{\gdef\theauthors{#1}}
\NewDocumentCommand{\professor}{m}{\gdef\theprofessor{#1}}
\NewDocumentCommand{\semester}{m}{\gdef\thesemester{#1}}

\renewcommand{\maketitle}{
    \begin{titlepage}
        \begin{center}
            {\scshape Universidade Federal do Rio Grande do Sul\\Instituto de Informática} \\
            \vspace*{\fill}
            {\large \thecourse} \\[24pt]
            {\Huge \thetitle} \\[48pt]
            \theauthors \\
            \vspace*{\fill}
            \theprofessor \\[2pt]
            \thesemester
        \end{center}
    \end{titlepage}
}

\course{Fundamentos de Banco de Dados – INF01145}
\title{
    Trabalho Prático – Etapa 2 \\[8pt]
    \Large Comandos de Definição de Dados (DDL)
}
\authors{
    Bruno Samuel Ardenghi Gonçalves — 550452 \\[2pt]
    Leonardo Azzi Martins — 323721
}
\professor{Prof.ª Karin Becker}
\semester{2025/1}

\begin{document}

\maketitle

\section{Entidades}

\subsection{Tabela \texttt{series}}
Uma série de livros contendo uma ou mais obras primárias explicitamente ordenadas e, opcionalmente, outras obras sem uma posição específica dentro da série.
\begin{lstlisting}
CREATE TABLE series (
    id SERIAL PRIMARY KEY,
    title VARCHAR(100) NOT NULL
);
\end{lstlisting}

\subsection{Tabela \texttt{works}}
Uma obra literária que engloba todas as suas edições.
\begin{lstlisting}
CREATE TABLE works (
    id SERIAL PRIMARY KEY,
    title VARCHAR(100) NOT NULL,
    first_publication_date DATE NOT NULL,
    mean_rating NUMERIC(2) CHECK (
        mean_rating BETWEEN 1 AND 5
    )
);
\end{lstlisting}

\subsection{Tabela \texttt{genres}}
Um gênero ou categoria de livros. É definido pelo seu nome e usado para agrupar obras com características similares.
\begin{lstlisting}
CREATE TABLE genres (
    id SERIAL PRIMARY KEY,
    slug VARCHAR(50) NOT NULL UNIQUE,
    label VARCHAR(50) NOT NULL,
    description VARCHAR(2000)
);
\end{lstlisting}

\newpage

\subsection{Tabela \texttt{authors}}
Um autor com zero ou mais obras atribuídas a si. Usado para agrupar livros escritos pela mesma pessoa e fornecer informações sobre estas através de um perfil.
\begin{lstlisting}
CREATE TABLE authors (
    id SERIAL PRIMARY KEY,
    name VARCHAR(150) NOT NULL,
    picture_url VARCHAR(500),
    birth_place VARCHAR(50),
    birth_date DATE,
    death_date DATE,
    website VARCHAR(500),
    biography VARCHAR(2000)
);
\end{lstlisting}

\subsection{Tabela \texttt{editions}}
Uma edição ou instância de uma obra. Também implementa o relacionamento \textit{Publication}, que atribui uma edição à obra a qual ela corresponde.
\begin{lstlisting}
CREATE TABLE editions (
    id SERIAL PRIMARY KEY,
    title VARCHAR(150) NOT NULL,
    page_count SMALLINT NOT NULL,
    format VARCHAR(25) NOT NULL,
    publication_date DATE NOT NULL,
    publisher VARCHAR(50) NOT NULL,
    language VARCHAR(50) NOT NULL,
    cover_image_url VARCHAR(500),
    summary VARCHAR(2000),
    isbn CHAR(13),
    asin CHAR(10),
    -- Relacionamento Publication
    work_id INTEGER NOT NULL REFERENCES works(id)
);
\end{lstlisting}

\newpage

\subsection{Tabela \texttt{users}}
Um usuário representa um perfil na plataforma, necessário para interagir com o site.
\begin{lstlisting}
CREATE TABLE users (
    id SERIAL PRIMARY KEY,
    email VARCHAR(254) NOT NULL UNIQUE,
    password VARCHAR(35) NOT NULL,
    first_name VARCHAR(50) NOT NULL,
    last_name VARCHAR(50),
    picture_url VARCHAR(500)
);
\end{lstlisting}

\subsection{Tabela \texttt{groups}}
Um grupo criado e composto por usuários, que podem ser moderadores ou membros.
\begin{lstlisting}
CREATE TABLE groups (
    id SERIAL PRIMARY KEY,
    name VARCHAR(100) NOT NULL,
    description VARCHAR(1000),
    picture_url VARCHAR(500)
);
\end{lstlisting}

\subsection{Tabela \texttt{lists}}
Uma lista de livros pública. Qualquer usuário pode adicionar livros à lista, assim como votar em obras já listadas.
\begin{lstlisting}
CREATE TABLE lists (
    id SERIAL PRIMARY KEY,
    title VARCHAR(100) NOT NULL,
    description VARCHAR(1000)
);
\end{lstlisting}

\newpage

\subsection{Tabela \texttt{list\_entries}}
Representa uma obra específica dentro de uma lista, implementando também o relacionamento \textit{Entry}, que a atribui à sua respectiva lista.
\begin{lstlisting}
CREATE TABLE list_entries (
    id SERIAL PRIMARY KEY,
    vote_count INTEGER NOT NULL CHECK (vote_count >= 0),
    -- Relacionamento Entry
    list_id INTEGER NOT NULL REFERENCES lists(id),
    edition_id INTEGER NOT NULL REFERENCES editions(id)
);
\end{lstlisting}

\subsection{Tabela \texttt{shelves}}
Uma estante ou coleção de livros pública de um usuário.
\begin{lstlisting}
CREATE TABLE shelves (
    id SERIAL PRIMARY KEY,
    slug VARCHAR(50) NOT NULL UNIQUE
);
\end{lstlisting}

\subsection{Tabela \texttt{quotes}}
Uma citação atribuída a um autor por um usuário. Implementa o relacionamento \textit{Attribution}, que formaliza a autoria da citação.
\begin{lstlisting}
CREATE TABLE quotes (
    id SERIAL PRIMARY KEY,
    quote VARCHAR(500) NOT NULL,
    -- Relacionamento Attribution
    author_id INTEGER NOT NULL REFERENCES authors(id)
);
\end{lstlisting}

\clearpage
\section{Relacionamentos}

\subsection{Tabela \texttt{positionings}}
Tabela de relacionamento que atribui uma obra a uma série, com ou sem uma posição específica.
\begin{lstlisting}
CREATE TABLE positionings (
    series_id INTEGER NOT NULL REFERENCES series(id),
    work_id INTEGER NOT NULL REFERENCES works(id),
    position NUMERIC(2),
    PRIMARY KEY (series_id, work_id)
);
\end{lstlisting}

\subsection{Tabela \texttt{categorizations}}
Tabela de relacionamento que atribui uma obra a um gênero.
\begin{lstlisting}
CREATE TABLE categorizations (
    work_id INTEGER NOT NULL REFERENCES works(id),
    genre_id INTEGER NOT NULL REFERENCES genres(id),
    PRIMARY KEY (work_id, genre_id)
);
\end{lstlisting}

\subsection{Tabela \texttt{authorships}}
Tabela de relacionamento que atribui uma obra aos seus autores.
\begin{lstlisting}
CREATE TABLE authorships (
    work_id INTEGER NOT NULL REFERENCES works(id),
    author_id INTEGER NOT NULL REFERENCES authors(id),
    PRIMARY KEY (work_id, author_id)
);
\end{lstlisting}

\newpage

\subsection{Tabela \texttt{storages}}
Tabela de relacionamento que atribui uma edição de um livro a uma estante de um usuário.
\begin{lstlisting}
CREATE TABLE storages (
    edition_id INTEGER NOT NULL REFERENCES editions(id),
    shelf_id INTEGER NOT NULL REFERENCES shelves(id),
    PRIMARY KEY (edition_id, shelf_id)
);
\end{lstlisting}

\subsection{Tabela \texttt{ownerships}}
Tabela de relacionamento que atribui uma estante ao seu usuário proprietário.
\begin{lstlisting}
CREATE TABLE ownerships (
    shelf_id INTEGER NOT NULL REFERENCES shelves(id),
    user_id INTEGER NOT NULL REFERENCES users(id),
    PRIMARY KEY (shelf_id, user_id)
);
\end{lstlisting}

\subsection{Tipo \texttt{tracking\_status}}
Define os possíveis status de rastreamento de leitura de um livro por um usuário.
\begin{lstlisting}
CREATE TYPE tracking_status AS ENUM ('to-read', 'currently-reading', 'read');
\end{lstlisting}

\newpage

\subsection{Tabela \texttt{trackings}}
Guarda o status e progresso de leitura de um usuário em relação a um livro.
\begin{lstlisting}
CREATE TABLE trackings (
    edition_id INTEGER NOT NULL REFERENCES editions(id),
    user_id INTEGER NOT NULL REFERENCES users(id),
    status tracking_status NOT NULL,
    progress SMALLINT DEFAULT 0,
    review VARCHAR(18800),
    reading_period DATERANGE,
    rating SMALLINT CHECK (
        rating BETWEEN 1 AND 5
    ),
    PRIMARY KEY (edition_id, user_id)
);
\end{lstlisting}

\subsection{Tabela \texttt{likes}}
Relacionamento criado quando um usuário curte uma citação.
\begin{lstlisting}
CREATE TABLE likes (
    quote_id INTEGER NOT NULL REFERENCES quotes(id),
    user_id INTEGER NOT NULL REFERENCES users(id),
    PRIMARY KEY (quote_id, user_id)
);
\end{lstlisting}

\subsection{Tabela \texttt{memberships}}
Relacionamento que atribui um usuário a um grupo do qual ele faz parte.
\begin{lstlisting}
CREATE TABLE memberships (
    user_id INTEGER NOT NULL REFERENCES users(id),
    group_id INTEGER NOT NULL REFERENCES groups(id),
    PRIMARY KEY (user_id, group_id)
);
\end{lstlisting}

\newpage

\subsection{Tabela \texttt{moderations}}
Relacionamento que atribui a posição de moderador de um grupo a um usuário.
\begin{lstlisting}
CREATE TABLE moderations (
    user_id INTEGER NOT NULL REFERENCES users(id),
    group_id INTEGER NOT NULL REFERENCES groups(id),
    PRIMARY KEY (user_id, group_id)
);
\end{lstlisting}

\subsection{Tabela \texttt{currently\_readings}}
Relacionamento que atribui um livro como leitura atual de um grupo.
\begin{lstlisting}
CREATE TABLE currently_readings (
    edition_id INTEGER NOT NULL REFERENCES editions(id),
    group_id INTEGER NOT NULL REFERENCES groups(id),
    start_date DATE,
    finish_date DATE,
    PRIMARY KEY (edition_id, group_id)
);
\end{lstlisting}

\subsection{Tabela \texttt{votes}}
Relacionamento criado quando um usuário vota em um livro em uma lista.
\begin{lstlisting}
CREATE TABLE votes (
    user_id INTEGER NOT NULL REFERENCES users(id),
    list_entry_id INTEGER NOT NULL REFERENCES list_entries(id),
    PRIMARY KEY (user_id, list_entry_id)
);
\end{lstlisting}

\end{document}