\documentclass{article}

\usepackage{geometry}
\usepackage{titling}
\usepackage{amsmath}
\usepackage{amssymb}
\usepackage{amsfonts}
\usepackage{amsthm}
\usepackage{stmaryrd}
\usepackage{enumitem}
\usepackage{xparse}

% -----------------------------------------------------------------------------

\NewDocumentCommand{\institute}{m}{\gdef\theinstitute{#1}}
\NewDocumentCommand{\course}{m}{\gdef\thecourse{#1}}
\NewDocumentCommand{\professor}{m}{\gdef\theprofessor{#1}}

\renewcommand{\maketitle}{
    \begin{titlepage}
        \begin{center}
            \large{\scshape{\theinstitute}} \\
            \vspace*{\fill}
            \Large{\thecourse} \\[2pt]
            \Huge{\thetitle} \\[36pt]
            \large{\theauthor} \\
            \vspace*{\fill}
            \large{\theprofessor} \\[2pt]
            \large{\thedate}
        \end{center}
    \end{titlepage}
}

\NewDocumentCommand{\problemBox}{m m m}{
    \noindent\fbox{
        \begin{minipage}{0.95\linewidth}
            \noindent$\mathbf{#1}$ \\[4pt]
            \textbf{Entrada}: #2 \\[2pt]
            \textbf{Pergunta}: #3
        \end{minipage}
    }
}

% -----------------------------------------------------------------------------

\geometry{
    paper=a4paper,
    top=2.5cm,
    bottom=3cm,
    left=2.5cm,
    right=2.5cm,
    headheight=14pt,
    footskip=1.5cm,
    headsep=1.2cm
}

\setlength{\fboxsep}{6pt}

\institute{
    Universidade Federal do Rio Grande do Sul \\
    Instituto de Informática
}

\course{Teoria da Computação N - INF05501}

\professor{Prof. Rodrigo Machado}

\title{Trabalho 4 - Redução de Problemas}

\author{
    Bruno Samuel Ardenghi Gonçalves — 550452 \\[2pt]
    João Pedro Müller Alvarenga — 577252 \\[2pt]
    Matheus Luís de Castro — 314943
}

\date{2024/2}

%------------------------------------------------------------------------------

\begin{document}

\maketitle

%------------------------------------------------------------------------------
%    1
%------------------------------------------------------------------------------

\section{}

\vspace{6pt}

\problemBox
{PP}
{um par $(M, w)$, sendo $M$ uma máquina de Turing sobre o
alfabeto $\Sigma$, e $w \in \Sigma^{*}$ uma palavra de entrada para $M$.}
{$w \in (\mathrm{ACEITA}(M) \cup \mathrm{REJEITA}(M))$?}

\vspace{8pt}

\problemBox
{PAT}
{uma máquina de Turing $M$.}
{$\mathrm{ACEITA}(M) = \Sigma^{*}$?}

\vspace{10pt}

\newtheorem*{theorem}{Teorema}
\begin{theorem}
    $\mathrm{PAT}$ é indecidível.
\end{theorem}

\renewcommand{\proofname}{\textbf{Demonstração}}
\begin{proof}
    Vamos construir uma redução $r : \mathrm{PP} \Rightarrow
    \mathrm{PAT}$ tal que, dada uma instância $(M, w)$ de
    $\mathrm{PP}$, produzimos uma nova máquina de Turing $M'$ como
    entrada para $\mathrm{PAT}$. A redução deve satisfazer:

    $$(M, w) \in \mathcal{Y}(\mathrm{PP})
    \iff M' \in \mathcal{Y}(\mathrm{PAT})$$.

    \noindent A máquina $M'$ opera da seguinte forma para uma entrada $t$:

    \begin{enumerate}[itemsep={2pt}, topsep={4pt}, parsep={0pt},
            label={\textbf{\arabic{*}.}}]
        \item Simula $M$ com entrada $w$:
            \begin{itemize}
                \item Se $M$ para (aceitando ou rejeitando), aceita.
                \item Se $M$ não para, $M'$ entra em loop infinito.
            \end{itemize}
    \end{enumerate}

    \vspace{8pt}

    \noindent Agora mostramos que $r$ é uma redução correta:

    \begin{itemize}[itemsep={2pt}, topsep={4pt}]
        \item \textbf{Caso 1:} Se $(M, w) \in
            \mathcal{Y}(\mathrm{PP})$, ou seja, $M$ para com $w$,
            então $M'$ aceita qualquer $t \in \Sigma^{*}$. Assim,
            $\mathrm{ACEITA}(M') = \Sigma^{*}$ e, portanto, $M' \in
            \mathcal{Y}(\mathrm{PAT})$.
        \item \textbf{Caso 2:} Se $(M, w) \in
            \mathcal{N}(\mathrm{PP})$, ou seja, $M$ não para com
            $w$, então $M'$ entra em loop infinito para qualquer $t
            \in \Sigma^{*}$. Assim, $\mathrm{ACEITA}(M') = \emptyset
            \neq \Sigma^{*}$ e, portanto, $M' \in \mathcal{N}(\mathrm{PAT})$.
    \end{itemize}

    \vspace{8pt}

    \noindent Como a redução $r$ está corretamente definida e sabemos
    que $\mathrm{PP}$ é
    indecidível, segue que $\mathrm{PAT}$ também é indecidível.
\end{proof}

%------------------------------------------------------------------------------
%    2
%------------------------------------------------------------------------------

\newpage

\section{}

\vspace{6pt}

\problemBox
{PP}
{um par $(M, w)$, sendo $M$ uma máquina de Turing sobre o
alfabeto $\Sigma$, e $w \in \Sigma^{*}$ uma palavra de entrada para $M$.}
{$w \in (\mathrm{ACEITA}(M) \cup \mathrm{REJEITA}(M))$?}

\vspace{8pt}

\problemBox
{PAPV}
{uma máquina de Turing $M$ sobre alfabeto
$\Sigma$.}
{$\varepsilon \in \mathrm{ACEITA}(M)$?}

\vspace{10pt}

\begin{theorem}
    $\mathrm{PAPV}$ é indecidível.
\end{theorem}

\renewcommand{\proofname}{\textbf{Demonstração}}
\begin{proof}
    Vamos construir uma redução $r : \mathrm{PP} \Rightarrow
    \mathrm{PAPV}$ tal que, dada uma instância $(M, w)$ de
    $\mathrm{PP}$, produzimos uma nova máquina de Turing $M'$ como
    entrada para $\mathrm{PAPV}$. A redução deve satisfazer:

    $$(M, w) \in \mathcal{Y}(\mathrm{PP})
    \iff M' \in \mathcal{Y}(\mathrm{PAPV})$$.

    \noindent A máquina $M'$ opera da seguinte forma para uma entrada $t$:

    \begin{enumerate}[itemsep={2pt}, topsep={4pt}, parsep={0pt},
            label={\textbf{\arabic{*}.}}]
        \item Simula $M$ com entrada $w$:
            \begin{itemize}
                \item Se $M$ para (aceitando ou rejeitando), aceita.
                \item Se $M$ não para, $M'$ entra em loop infinito.
            \end{itemize}
    \end{enumerate}

    \vspace{8pt}

    \noindent Agora mostramos que $r$ é uma redução correta:

    \begin{itemize}[itemsep={2pt}, topsep={4pt}]
        \item \textbf{Caso 1:} Se $(M, w) \in
            \mathcal{Y}(\mathrm{PP})$, ou seja, $M$ para com $w$,
            então $M'$ aceita qualquer $t \in \Sigma^{*}$, incluindo
            $\varepsilon$. Assim, $\varepsilon \in
            \mathrm{ACEITA}(M')$ e, portanto, $M' \in
            \mathcal{Y}(\mathrm{PAPV})$.
        \item \textbf{Caso 2:} Se $(M, w) \in
            \mathcal{N}(\mathrm{PP})$, ou seja, $M$ não para com
            $w$, então $M'$ entra em loop infinito para qualquer $t
            \in \Sigma^{*}$, incluindo $\varepsilon$. Assim,
            $\varepsilon \notin \mathrm{ACEITA}(M')$ e, portanto, $M'
            \in \mathcal{N}(\mathrm{PAPV})$.
    \end{itemize}

    \vspace{8pt}

    \noindent Como a redução $r$ está corretamente definida e sabemos
    que $\mathrm{PP}$ é
    indecidível, segue que $\mathrm{PAPV}$ também é indecidível.
\end{proof}

%------------------------------------------------------------------------------
%    3
%-----------------------------------------------------------------------------

\newpage

\section{}

\vspace{6pt}

\problemBox
{PP}
{um par $(M, w)$, sendo $M$ uma máquina de Turing sobre o
alfabeto $\Sigma$, e $w \in \Sigma^{*}$ uma palavra de entrada para $M$.}
{$w \in (\mathrm{ACEITA}(M) \cup \mathrm{REJEITA}(M))$?}

\vspace{8pt}

\problemBox
{PLAU}
{uma máquina de Turing $M$ sobre alfabeto $\Sigma$.}
{$\mathrm{ACEITA}(M) = \{w\}$ (conjunto unitário contendo $w$), para
um $w \in \Sigma^{*}$ qualquer?}

\vspace{10pt}

\begin{theorem}
    $\mathrm{PLAU}$ é indecidível.
\end{theorem}

\renewcommand{\proofname}{\textbf{Demonstração}}
\begin{proof}
    Vamos construir uma redução $r : \mathrm{PP} \Rightarrow
    \mathrm{PLAU}$ tal que, dada uma instância $(M, w)$ de
    $\mathrm{PP}$, produzimos uma nova máquina de Turing $M'$ como
    entrada para $\mathrm{PLAU}$. A redução deve satisfazer:

    $$(M, w) \in \mathcal{Y}(\mathrm{PP})
    \iff M' \in \mathcal{Y}(\mathrm{PLAU})$$.

    \noindent A máquina $M'$ opera da seguinte forma para uma entrada $t$:

    \begin{enumerate}[itemsep={2pt}, topsep={4pt}, parsep={0pt},
            label={\textbf{\arabic{*}.}}]
        \item Verifica se $t = w$:
            \begin{itemize}
                \item Se $t = w$, continua.
                \item Se $t \neq w$, rejeita.
            \end{itemize}
        \item Simula $M$ com entrada $w$:
            \begin{itemize}
                \item Se $M$ para (aceitando ou rejeitando), aceita.
                \item Se $M$ não para, $M'$ entra em loop infinito.
            \end{itemize}
    \end{enumerate}

    \vspace{8pt}

    \noindent Agora mostramos que $r$ é uma redução correta:

    \begin{itemize}[itemsep={2pt}, topsep={4pt}]
        \item \textbf{Caso 1:} Se $(M, w) \in
            \mathcal{Y}(\mathrm{PP})$, ou seja, $M$ para com $w$,
            então $M'$ aceita $t$ se, e somente se, $t = w$. Assim,
            $\mathrm{ACEITA}(M') = \{w\}$ e, portanto, $M' \in
            \mathcal{Y}(\mathrm{PLAU})$.

        \item \textbf{Caso 2:} Se $(M, w) \in
            \mathcal{N}(\mathrm{PP})$, ou seja, $M$ não para com $w$,
            então $M'$ rejeita para entradas diferentes de $w$ ou
            entra em loop infinito para $t = w$. Assim,
            $\mathrm{ACEITA}(M') = \emptyset$ e, portanto, $M' \in
            \mathcal{N}(\mathrm{PLAU})$.
    \end{itemize}

    \vspace{8pt}

    \noindent Como a redução $r$ está corretamente definida e sabemos
    que $\mathrm{PP}$ é
    indecidível, segue que $\mathrm{PLAU}$ também é indecidível.
\end{proof}

%------------------------------------------------------------------------------

\end{document}
